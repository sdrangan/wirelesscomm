\documentclass[11pt]{article}

\usepackage{fullpage}
\usepackage{amsmath, amssymb, bm, cite, epsfig, psfrag}
\usepackage{graphicx}
\usepackage{float}
\usepackage{amsthm}
\usepackage{amsfonts}
\usepackage{listings}
\usepackage{cite}
\usepackage{hyperref}
\usepackage{tikz}
\usepackage{enumitem}
\usetikzlibrary{shapes,arrows}
\usepackage{mdframed}
\usepackage{mcode}
\usepackage{siunitx}
\usepackage{mathtools}
%\usetikzlibrary{dsp,chains}

%\restylefloat{figure}
%\theoremstyle{plain}      \newtheorem{theorem}{Theorem}
%\theoremstyle{definition} \newtheorem{definition}{Definition}

\def\del{\partial}
\def\ds{\displaystyle}
\def\ts{\textstyle}
\def\beq{\begin{equation}}
\def\eeq{\end{equation}}
\def\beqa{\begin{eqnarray}}
\def\eeqa{\end{eqnarray}}
\def\beqan{\begin{eqnarray*}}
\def\eeqan{\end{eqnarray*}}
\def\nn{\nonumber}
\def\binomial{\mathop{\mathrm{binomial}}}
\def\half{{\ts\frac{1}{2}}}
\def\Half{{\frac{1}{2}}}
\def\N{{\mathbb{N}}}
\def\Z{{\mathbb{Z}}}
\def\Q{{\mathbb{Q}}}
\def\R{{\mathbb{R}}}
\def\C{{\mathbb{C}}}
\def\argmin{\mathop{\mathrm{arg\,min}}}
\def\argmax{\mathop{\mathrm{arg\,max}}}
%\def\span{\mathop{\mathrm{span}}}
\def\diag{\mathop{\mathrm{diag}}}
\def\x{\times}
\def\limn{\lim_{n \rightarrow \infty}}
\def\liminfn{\liminf_{n \rightarrow \infty}}
\def\limsupn{\limsup_{n \rightarrow \infty}}
\def\MID{\,|\,}
\def\MIDD{\,;\,}

\newtheorem{proposition}{Proposition}
\newtheorem{definition}{Definition}
\newtheorem{theorem}{Theorem}
\newtheorem{lemma}{Lemma}
\newtheorem{corollary}{Corollary}
\newtheorem{assumption}{Assumption}
\newtheorem{claim}{Claim}
\def\qed{\mbox{} \hfill $\Box$}
\setlength{\unitlength}{1mm}

\def\bhat{\widehat{b}}
\def\ehat{\widehat{e}}
\def\phat{\widehat{p}}
\def\qhat{\widehat{q}}
\def\rhat{\widehat{r}}
\def\shat{\widehat{s}}
\def\uhat{\widehat{u}}
\def\ubar{\overline{u}}
\def\vhat{\widehat{v}}
\def\xhat{\widehat{x}}
\def\xbar{\overline{x}}
\def\zhat{\widehat{z}}
\def\zbar{\overline{z}}
\def\la{\leftarrow}
\def\ra{\rightarrow}
\def\MSE{\mbox{\small \sffamily MSE}}
\def\SNR{\mbox{\small \sffamily SNR}}
\def\SINR{\mbox{\small \sffamily SINR}}
\def\arr{\rightarrow}
\def\Exp{\mathbb{E}}
\def\var{\mbox{var}}
\def\Tr{\mbox{Tr}}
\def\tm1{t\! - \! 1}
\def\tp1{t\! + \! 1}

\def\Xset{{\cal X}}

\newcommand{\one}{\mathbf{1}}
\newcommand{\abf}{\mathbf{a}}
\newcommand{\bbf}{\mathbf{b}}
\newcommand{\dbf}{\mathbf{d}}
\newcommand{\ebf}{\mathbf{e}}
\newcommand{\gbf}{\mathbf{g}}
\newcommand{\hbf}{\mathbf{h}}
\newcommand{\pbf}{\mathbf{p}}
\newcommand{\pbfhat}{\widehat{\mathbf{p}}}
\newcommand{\qbf}{\mathbf{q}}
\newcommand{\qbfhat}{\widehat{\mathbf{q}}}
\newcommand{\rbf}{\mathbf{r}}
\newcommand{\rbfhat}{\widehat{\mathbf{r}}}
\newcommand{\sbf}{\mathbf{s}}
\newcommand{\sbfhat}{\widehat{\mathbf{s}}}
\newcommand{\ubf}{\mathbf{u}}
\newcommand{\ubfhat}{\widehat{\mathbf{u}}}
\newcommand{\utildebf}{\tilde{\mathbf{u}}}
\newcommand{\vbf}{\mathbf{v}}
\newcommand{\vbfhat}{\widehat{\mathbf{v}}}
\newcommand{\wbf}{\mathbf{w}}
\newcommand{\wbfhat}{\widehat{\mathbf{w}}}
\newcommand{\xbf}{\mathbf{x}}
\newcommand{\xbfhat}{\widehat{\mathbf{x}}}
\newcommand{\xbfbar}{\overline{\mathbf{x}}}
\newcommand{\ybf}{\mathbf{y}}
\newcommand{\zbf}{\mathbf{z}}
\newcommand{\zbfbar}{\overline{\mathbf{z}}}
\newcommand{\zbfhat}{\widehat{\mathbf{z}}}
\newcommand{\Ahat}{\widehat{A}}
\newcommand{\Abf}{\mathbf{A}}
\newcommand{\Bbf}{\mathbf{B}}
\newcommand{\Cbf}{\mathbf{C}}
\newcommand{\Bbfhat}{\widehat{\mathbf{B}}}
\newcommand{\Dbf}{\mathbf{D}}
\newcommand{\Ebf}{\mathbf{E}}
\newcommand{\Gbf}{\mathbf{G}}
\newcommand{\Hbf}{\mathbf{H}}
\newcommand{\Ibf}{\mathbf{I}}
\newcommand{\Kbf}{\mathbf{K}}
\newcommand{\Pbf}{\mathbf{P}}
\newcommand{\Phat}{\widehat{P}}
\newcommand{\Qbf}{\mathbf{Q}}
\newcommand{\Rbf}{\mathbf{R}}
\newcommand{\Rhat}{\widehat{R}}
\newcommand{\Sbf}{\mathbf{S}}
\newcommand{\Ubf}{\mathbf{U}}
\newcommand{\Vbf}{\mathbf{V}}
\newcommand{\Wbf}{\mathbf{W}}
\newcommand{\Xhat}{\widehat{X}}
\newcommand{\Xbf}{\mathbf{X}}
\newcommand{\Ybf}{\mathbf{Y}}
\newcommand{\Zbf}{\mathbf{Z}}
\newcommand{\Zhat}{\widehat{Z}}
\newcommand{\Zbfhat}{\widehat{\mathbf{Z}}}
\def\alphabf{{\boldsymbol \alpha}}
\def\betabf{{\boldsymbol \beta}}
\def\mubf{{\boldsymbol \mu}}
\def\lambdabf{{\boldsymbol \lambda}}
\def\etabf{{\boldsymbol \eta}}
\def\xibf{{\boldsymbol \xi}}
\def\taubf{{\boldsymbol \tau}}
\def\sigmahat{{\widehat{\sigma}}}
\def\thetabf{{\bm{\theta}}}
\def\thetabfhat{{\widehat{\bm{\theta}}}}
\def\thetahat{{\widehat{\theta}}}
\def\mubar{\overline{\mu}}
\def\muavg{\mu}
\def\sigbf{\bm{\sigma}}
\def\etal{\emph{et al.}}
\def\Ggothic{\mathfrak{G}}
\def\Pset{{\mathcal P}}
\newcommand{\bigCond}[2]{\bigl({#1} \!\bigm\vert\! {#2} \bigr)}
\newcommand{\BigCond}[2]{\Bigl({#1} \!\Bigm\vert\! {#2} \Bigr)}

\def\Rect{\mathop{Rect}}
\def\sinc{\mathop{sinc}}
\def\NF{\mathrm{NF}}
\def\Real{\mathrm{Re}}
\def\Imag{\mathrm{Im}}
\newcommand{\tran}{^{\text{\sf T}}}
\newcommand{\herm}{^{\text{\sf H}}}


% Solution environment
\definecolor{lightgray}{gray}{0.95}
\newmdenv[linecolor=white,backgroundcolor=lightgray,frametitle=Solution:]{solution}



\begin{document}

\title{Problems:  Channel Estimation and Equalization\\
ECE-GY 6023. Wireless Communications}
\author{Prof.\ Sundeep Rangan}
\date{}

\maketitle


\begin{enumerate}

\item \emph{Kernel regression:}  
Consider a kernel regression estimate of the form
\begin{equation} \label{eq:kernel}
    \widehat{h}[n] = \frac{\sum_{\ell \in I} w_\ell \widehat{h}_0[n+\ell]}
    {\sum_{\ell \in I} w_\ell},
\end{equation}
where $I$ is a set of reference symbol locations, $\widehat{h}_0[n]$ are raw estimates of the
channel on the reference symbols and $w_\ell$ is a kernel.  
Suppose you use a triangular kernel,
\[
    w_\ell = \max\{ 1-\ell/L, 0 \},
\]
for some width $L$ and the reference symbols are spaced every $d$ positions,
\[
    I = \{0, d, 2d, \quad, Md\}.
\]
Assume $L=12$, $d=4$ and $M=10$.  For each value $n$, below write $\widehat{h}[n]$ as a linear
combination of the values $\widehat{h}_0[k]$.
\begin{enumerate}[label=(\alph*)]
\item The symbol on the edge, $n=0$.
\item A symbol in the middle located on a reference symbol location,  $n=20$.
\item A symbol in the middle located between two reference symbol locations, $n=22$.
\end{enumerate}

\item \emph{Channel estimation error for a stochastic model:}  Suppose that
we form an estimate,
\[
    \widehat{h}[n] = \frac{1}{2} \left[ \widehat{h}_0[n-L] + \widehat{h}_0[n+L] \right].
\]
with measurements
\[
    \widehat{h}_0[k] = h[k] + v[k], \quad v[k] \sim C{\mathcal N}(0,N_v).
\]
Suppose we can model $h[k]$ as a wide sense stationary Gaussian random process
with auto-correlation $R[k] = \Exp( h[n]h[n+k]^* )$.
\begin{enumerate}[label=(\alph*)]
\item Find the mean squared error,
\[
    \epsilon = \Exp | \widehat{h}[n] - h[n]|^2 
\]
in terms of the auto-correlation $R[n]$ and the error in the raw channel estimate $N_v$.

\item If $h[n]$ is a narrowband fading process with Jake's spectrum then its auto-correlation is
\[
    R[k] = E_s J_0(2\pi f_{\rm max} k T)
\]
where $E_s$ is the energy per sample, 
$f_{\rm max}$ is the maximum Doppler spread and $T$ is the sample period.
Write and plot the normalized MSE, 
$\epsilon/E_s$ as a function of $f_{\rm max}LT$ for the case when $N_v=0$.
Note that when there is zero noise, $\mathrm{MSE}$, represents the bias squared.

\item In the above model, there is one reference symbol every $2L$ samples,
so the overhead is $1/(2L)$.
Suppose that $f_{\rm max} = $\, \SI{100}{Hz}, and the sample period is $T=$\, \SI{1}{\micro\second}.  Using the model in part (b), 
what is the minimum overhead if we need the MSE, $\epsilon$,
to be less than \SI{20}{dB} below $E_s$?

\end{enumerate}


\item \emph{Joint Likelihoods}:  
Suppose that we have two BPSK symbols, $i=0,1$:
\[
    r_i = hx_i + w_i, \quad w_i \sim C{\mathcal N}(0,N_0),    
    \quad x_i = \pm 1,
\]
where the noise $w_i$ is i.i.d.  The channel gain $h$ is unknown and can
be modeled as a complex Gaussian $h \sim C{\mathcal N}(0,E_s)$.
We use the first symbol, $x_0=1$, as a reference symbol. So, there are
two possibilities:  
\[
    \xbf = \xbf^{(1)}=(1,1) \mbox{ or } \xbf^{(2)}=(1,-1).
\]
We will compute the LLR for these two possibilties.
\begin{enumerate}[label=(\alph*)]
\item What is the mean and covariance matrix of the vector $\rbf=(r_0,r_1)$
for the two values of $\xbf$?

\item Using the fact that $\rbf$ is a Gaussian random vector,
what is the log likelihood ratio
\[
    \mathrm{LLR} = \log \left[ \frac{p(\rbf|\xbf=(1,1))}{p(\rbf|\xbf=(1,-1))} \right].
\]
You can simplify the expression with the matrix identity,
\[
    (\Ibf + \gamma \ubf \ubf^*)^{-1} = \Ibf - \frac{\gamma}{1 + \gamma\|\ubf\|^2} \ubf\ubf^*
\]
for any vector $\ubf$.
\end{enumerate}


\end{enumerate}






\end{document}



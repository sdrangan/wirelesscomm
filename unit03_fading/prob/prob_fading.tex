\documentclass[11pt]{article}

\usepackage{fullpage}
\usepackage{amsmath, amssymb, bm, cite, epsfig, psfrag}
\usepackage{graphicx}
\usepackage{float}
\usepackage{amsthm}
\usepackage{amsfonts}
\usepackage{listings}
\usepackage{cite}
\usepackage{hyperref}
\usepackage{tikz}
\usepackage{enumitem}
\usetikzlibrary{shapes,arrows}
\usepackage{mdframed}
\usepackage{mcode}
\usepackage{siunitx}
%\usetikzlibrary{dsp,chains}

%\restylefloat{figure}
%\theoremstyle{plain}      \newtheorem{theorem}{Theorem}
%\theoremstyle{definition} \newtheorem{definition}{Definition}

\def\del{\partial}
\def\ds{\displaystyle}
\def\ts{\textstyle}
\def\beq{\begin{equation}}
\def\eeq{\end{equation}}
\def\beqa{\begin{eqnarray}}
\def\eeqa{\end{eqnarray}}
\def\beqan{\begin{eqnarray*}}
\def\eeqan{\end{eqnarray*}}
\def\nn{\nonumber}
\def\binomial{\mathop{\mathrm{binomial}}}
\def\half{{\ts\frac{1}{2}}}
\def\Half{{\frac{1}{2}}}
\def\N{{\mathbb{N}}}
\def\Z{{\mathbb{Z}}}
\def\Q{{\mathbb{Q}}}
\def\R{{\mathbb{R}}}
\def\C{{\mathbb{C}}}
\def\argmin{\mathop{\mathrm{arg\,min}}}
\def\argmax{\mathop{\mathrm{arg\,max}}}
%\def\span{\mathop{\mathrm{span}}}
\def\diag{\mathop{\mathrm{diag}}}
\def\x{\times}
\def\limn{\lim_{n \rightarrow \infty}}
\def\liminfn{\liminf_{n \rightarrow \infty}}
\def\limsupn{\limsup_{n \rightarrow \infty}}
\def\MID{\,|\,}
\def\MIDD{\,;\,}

\newtheorem{proposition}{Proposition}
\newtheorem{definition}{Definition}
\newtheorem{theorem}{Theorem}
\newtheorem{lemma}{Lemma}
\newtheorem{corollary}{Corollary}
\newtheorem{assumption}{Assumption}
\newtheorem{claim}{Claim}
\def\qed{\mbox{} \hfill $\Box$}
\setlength{\unitlength}{1mm}

\def\bhat{\widehat{b}}
\def\ehat{\widehat{e}}
\def\phat{\widehat{p}}
\def\qhat{\widehat{q}}
\def\rhat{\widehat{r}}
\def\shat{\widehat{s}}
\def\uhat{\widehat{u}}
\def\ubar{\overline{u}}
\def\vhat{\widehat{v}}
\def\xhat{\widehat{x}}
\def\xbar{\overline{x}}
\def\zhat{\widehat{z}}
\def\zbar{\overline{z}}
\def\la{\leftarrow}
\def\ra{\rightarrow}
\def\MSE{\mbox{\small \sffamily MSE}}
\def\SNR{\mbox{\small \sffamily SNR}}
\def\SINR{\mbox{\small \sffamily SINR}}
\def\arr{\rightarrow}
\def\Exp{\mathbb{E}}
\def\var{\mbox{var}}
\def\Tr{\mbox{Tr}}
\def\tm1{t\! - \! 1}
\def\tp1{t\! + \! 1}

\def\Xset{{\cal X}}

\newcommand{\one}{\mathbf{1}}
\newcommand{\abf}{\mathbf{a}}
\newcommand{\bbf}{\mathbf{b}}
\newcommand{\dbf}{\mathbf{d}}
\newcommand{\ebf}{\mathbf{e}}
\newcommand{\gbf}{\mathbf{g}}
\newcommand{\hbf}{\mathbf{h}}
\newcommand{\pbf}{\mathbf{p}}
\newcommand{\pbfhat}{\widehat{\mathbf{p}}}
\newcommand{\qbf}{\mathbf{q}}
\newcommand{\qbfhat}{\widehat{\mathbf{q}}}
\newcommand{\rbf}{\mathbf{r}}
\newcommand{\rbfhat}{\widehat{\mathbf{r}}}
\newcommand{\sbf}{\mathbf{s}}
\newcommand{\sbfhat}{\widehat{\mathbf{s}}}
\newcommand{\ubf}{\mathbf{u}}
\newcommand{\ubfhat}{\widehat{\mathbf{u}}}
\newcommand{\utildebf}{\tilde{\mathbf{u}}}
\newcommand{\vbf}{\mathbf{v}}
\newcommand{\vbfhat}{\widehat{\mathbf{v}}}
\newcommand{\wbf}{\mathbf{w}}
\newcommand{\wbfhat}{\widehat{\mathbf{w}}}
\newcommand{\xbf}{\mathbf{x}}
\newcommand{\xbfhat}{\widehat{\mathbf{x}}}
\newcommand{\xbfbar}{\overline{\mathbf{x}}}
\newcommand{\ybf}{\mathbf{y}}
\newcommand{\zbf}{\mathbf{z}}
\newcommand{\zbfbar}{\overline{\mathbf{z}}}
\newcommand{\zbfhat}{\widehat{\mathbf{z}}}
\newcommand{\Ahat}{\widehat{A}}
\newcommand{\Abf}{\mathbf{A}}
\newcommand{\Bbf}{\mathbf{B}}
\newcommand{\Cbf}{\mathbf{C}}
\newcommand{\Bbfhat}{\widehat{\mathbf{B}}}
\newcommand{\Dbf}{\mathbf{D}}
\newcommand{\Ebf}{\mathbf{E}}
\newcommand{\Gbf}{\mathbf{G}}
\newcommand{\Hbf}{\mathbf{H}}
\newcommand{\Kbf}{\mathbf{K}}
\newcommand{\Pbf}{\mathbf{P}}
\newcommand{\Phat}{\widehat{P}}
\newcommand{\Qbf}{\mathbf{Q}}
\newcommand{\Rbf}{\mathbf{R}}
\newcommand{\Rhat}{\widehat{R}}
\newcommand{\Sbf}{\mathbf{S}}
\newcommand{\Ubf}{\mathbf{U}}
\newcommand{\Vbf}{\mathbf{V}}
\newcommand{\Wbf}{\mathbf{W}}
\newcommand{\Xhat}{\widehat{X}}
\newcommand{\Xbf}{\mathbf{X}}
\newcommand{\Ybf}{\mathbf{Y}}
\newcommand{\Zbf}{\mathbf{Z}}
\newcommand{\Zhat}{\widehat{Z}}
\newcommand{\Zbfhat}{\widehat{\mathbf{Z}}}
\def\alphabf{{\boldsymbol \alpha}}
\def\betabf{{\boldsymbol \beta}}
\def\mubf{{\boldsymbol \mu}}
\def\lambdabf{{\boldsymbol \lambda}}
\def\etabf{{\boldsymbol \eta}}
\def\xibf{{\boldsymbol \xi}}
\def\taubf{{\boldsymbol \tau}}
\def\sigmahat{{\widehat{\sigma}}}
\def\thetabf{{\bm{\theta}}}
\def\thetabfhat{{\widehat{\bm{\theta}}}}
\def\thetahat{{\widehat{\theta}}}
\def\mubar{\overline{\mu}}
\def\muavg{\mu}
\def\sigbf{\bm{\sigma}}
\def\etal{\emph{et al.}}
\def\Ggothic{\mathfrak{G}}
\def\Pset{{\mathcal P}}
\newcommand{\bigCond}[2]{\bigl({#1} \!\bigm\vert\! {#2} \bigr)}
\newcommand{\BigCond}[2]{\Bigl({#1} \!\Bigm\vert\! {#2} \Bigr)}

\def\Rect{\mathop{Rect}}
\def\sinc{\mathop{sinc}}
\def\NF{\mathrm{NF}}
\def\Real{\mathrm{Re}}
\def\Imag{\mathrm{Im}}
\newcommand{\tran}{^{\text{\sf T}}}
\newcommand{\herm}{^{\text{\sf H}}}


% Solution environment
\definecolor{lightgray}{gray}{0.95}
\newmdenv[linecolor=white,backgroundcolor=lightgray,frametitle=Solution:]{solution}



\begin{document}

\title{Problems:  Small-Scale Fading\\
ECE-GY 6023. Wireless Communications}
\author{Prof.\ Sundeep Rangan}
\date{}

\maketitle


\begin{enumerate}

\item \emph{Doppler shift and frequency offset:}
\begin{enumerate}[label=(\alph*)]
\item Suppose that a vehicle is traveling at \SI{100}{km/h} directly towards a base station.
If the carrier frequency is \SI{2.5}{GHz}, what is the maximum Doppler?

\item Frequency shifts can also occur due to differences in the local oscillators
(LOs) at the TX and RX.  Suppose that the LO error is 1 part per million (ppm)
meaning that the frequency error is 1 millionth of the carrier.  What is the
frequency shift?
\end{enumerate}


\item \emph{Doppler and match filter}:  Suppose that we receive a signal
\[
    r(t) = u(t)e^{2\pi i ft} + w(t),
\]
where $u(t)$ is a transmitted signal, $f$ is a frequency shift
and $w(t)$ is WGN with PSD $N_0$.  Assume $|u(t)|^2=P$ for all $t$
for some received power level $P > 0$.
We then compute a matched filter,
\[
    z = \frac{1}{T} \int_0^T u(t)^*r(t)dt.
\]
\begin{enumerate}[label=(\alph*)]
\item Write the MF response as $z = x + v$ where $x$ is due to the signal
and $v$ is due to the noise.  Your expression for $x$ should have a sinc
function in it.

\item Write an expression for the SNR defined as
\[
    \SNR = \frac{|x|^2}{\Exp|v|^2}.
\]
The expression should be in terms of the integration time $T$,
SNR $P/N_0$ and frequency offset $f$.

\item When there is no frequency offset (i.e.\ $f=0$), the SNR increases
linearly with the integration time.  However, when there is an uncompensated
frequency offset, there is an optimal integration time beyond which
the SNR begins to drop.  Find the maximum SNR and optimal integration time.

The maximization will not have a closed-form answer.  So, we will use MATLAB.
Specifically, plot the SNR as a function of $T$ when $P/N_0=1$ and $f=1$.  Then,
write an expression to translate your answer to other values of $P/N_0$ and $f$.

\item Suppose that the frequency offset is $f=$ 100 Hz, the received power
is $P=$ -100 dBm and the noise power density is $N_0=$ -140 dBm/Hz.
What is the optimal integration time $T$ and maximum SNR?

\end{enumerate}

\item A received signal has two paths:
\begin{itemize}
\item Path 1:  power -100 dBm and Doppler shift 100 Hz,
\item Path 2:  power -103 dBm and Doppler shift -50 Hz.
\end{itemize}
\begin{enumerate}[label=(\alph*)]
\item Draw the signal power as a function of time.
\item What is the average receive power in dBm?
\item What is the time it takes to go from the maximum to minimum power?
\item What is the fraction of time the power is greater than -101 dBm?
\end{enumerate}

\item (*) Suppose that a narrowband complex channel $h(t)$ be modeled as
a wide-sense stationary random process.
\begin{enumerate}[label=(\alph*)]

\item Let $\rho(\tau)$ be the relative change
in $h(t)$ over a time $\tau$.
\[
    \rho(\tau) = \frac{\Exp|h(t)-h(t+\tau)|^2}{\Exp|h(t)|^2}
\]
Write $\rho(\tau)$ in terms of the autocorrelation function
$R(\tau) = \Exp h(t)h^*(t-\tau)$.

\item If $h(t)$ follows a Jakes' spectrum with maximum Doppler $f_{max} = $ 200 Hz,
what is the time it takes the channel to change by 10\%?
You will need MATLAB to evaluate the Bessel function.
\end{enumerate}

\item (*) Consider a multipath fading channel of the form
\[
    y(t) = \frac{1}{\sqrt{L}}\sum_{\ell = 1}^L g_\ell
        e^{2\pi if_\ell} x(t-\tau_\ell), \quad f_\ell = f_{max}\cos(\theta_\ell),
\]
where $L$ is the number of paths, and for each path $\ell$,
$g_\ell$ is its complex gain, $\theta_\ell$ the AoA, $f_\ell$ the Doppler shift
and $\tau_\ell$ the delay.  This channel has a time-varying frequency response
\beq \label{eq:Htf}
    H(t,f) = \frac{1}{\sqrt{L}}
    \sum_{\ell =1}^L g_\ell e^{2\pi  i(f_\ell t - \tau_\ell f)},
    \quad f_\ell = f_{max}\cos\theta_\ell.
\eeq
Suppose that we model the path parameters statistically.  That is,
we model $g_\ell,\theta_\ell,\tau_\ell$ as random variables that
are all independent
from one another and across different values of $\ell$.  Assume that the
distributions of the random variables are identical for different
path indices $\ell$.
In addition, assume that
$g_\ell$ are zero mean, with average power gain
$\Exp|g_\ell|^2 = G$ for some value $G > 0$.

\begin{enumerate}[label=(\alph*)]
\item Suppose that the delays are exponentially distributed with pdf,
\[
    p(\tau_\ell) = \frac{1}{\lambda}e^{-\tau_\ell/\lambda},
\]
for some $\lambda$ representing the delay spread.
What is the autocorrelation function in frequency
\[
    R(\delta f) = \Exp\left[ H(t,f)H^*(t,f-\Delta f) \right].
\]

\item Suppose we define the 3~dB coherence bandwidth as the frequency $W$
where the frequency response changes by less than 3~dB.  That is,
\[
    \Exp|H(t,f)-H(t,f+W)|^2 \leq \beta \Exp|H(t,f)|^2, \quad \beta = 0.5.
\]
Under the above exponential delay model,
if the delay spread is $\lambda =$ 100 ns, what is the 3~dB bandwidth?
\end{enumerate}

\item
\begin{enumerate}[label=(\alph*)]
\item (*) Consider the time-varying frequency response \eqref{eq:Htf}.
Suppose that the AoAs are uniformly distributed in $\theta_\ell \in [-\Delta/2,\Delta/2]$.
Write an expression for the autocorrelation over time,
\[
    R(\tau) := \Exp H(t,f)H^*(t-\tau,f).
\]
Your expression should have an integral.  Do not evaluate this integral.

\item In 3D channel models, the paths arrive with an \emph{elevation} (vertical)
angle $\omega_\ell$ to the $z$-axis
and \emph{azimuth} (horizontal) angle $\varphi_\ell$ to the $x$-axis on the
$xy$-plane.
Suppose that the receiver moves along the positive $x$-axis.  What is the Doppler
shift of a path arriving at angles $(\omega_\ell,\varphi_\ell)$.

\end{enumerate}

\end{enumerate}

\end{document}


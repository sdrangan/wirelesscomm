\documentclass[11pt]{article}

\usepackage{fullpage}
\usepackage{amsmath, amssymb, bm, cite, epsfig, psfrag}
\usepackage{graphicx}
\usepackage{float}
\usepackage{amsthm}
\usepackage{amsfonts}
\usepackage{listings}
\usepackage{cite}
\usepackage{hyperref}
\usepackage{tikz}
\usepackage{enumitem}
\usetikzlibrary{shapes,arrows}
\usepackage{mdframed}
\usepackage{mcode}
%\usetikzlibrary{dsp,chains}

%\restylefloat{figure}
%\theoremstyle{plain}      \newtheorem{theorem}{Theorem}
%\theoremstyle{definition} \newtheorem{definition}{Definition}

\def\del{\partial}
\def\ds{\displaystyle}
\def\ts{\textstyle}
\def\beq{\begin{equation}}
\def\eeq{\end{equation}}
\def\beqa{\begin{eqnarray}}
\def\eeqa{\end{eqnarray}}
\def\beqan{\begin{eqnarray*}}
\def\eeqan{\end{eqnarray*}}
\def\nn{\nonumber}
\def\binomial{\mathop{\mathrm{binomial}}}
\def\half{{\ts\frac{1}{2}}}
\def\Half{{\frac{1}{2}}}
\def\N{{\mathbb{N}}}
\def\Z{{\mathbb{Z}}}
\def\Q{{\mathbb{Q}}}
\def\R{{\mathbb{R}}}
\def\C{{\mathbb{C}}}
\def\argmin{\mathop{\mathrm{arg\,min}}}
\def\argmax{\mathop{\mathrm{arg\,max}}}
%\def\span{\mathop{\mathrm{span}}}
\def\diag{\mathop{\mathrm{diag}}}
\def\x{\times}
\def\limn{\lim_{n \rightarrow \infty}}
\def\liminfn{\liminf_{n \rightarrow \infty}}
\def\limsupn{\limsup_{n \rightarrow \infty}}
\def\GV{Guo and Verd{\'u}}
\def\MID{\,|\,}
\def\MIDD{\,;\,}

\newtheorem{proposition}{Proposition}
\newtheorem{definition}{Definition}
\newtheorem{theorem}{Theorem}
\newtheorem{lemma}{Lemma}
\newtheorem{corollary}{Corollary}
\newtheorem{assumption}{Assumption}
\newtheorem{claim}{Claim}
\def\qed{\mbox{} \hfill $\Box$}
\setlength{\unitlength}{1mm}

\def\bhat{\widehat{b}}
\def\ehat{\widehat{e}}
\def\phat{\widehat{p}}
\def\qhat{\widehat{q}}
\def\rhat{\widehat{r}}
\def\shat{\widehat{s}}
\def\uhat{\widehat{u}}
\def\ubar{\overline{u}}
\def\vhat{\widehat{v}}
\def\xhat{\widehat{x}}
\def\xbar{\overline{x}}
\def\zhat{\widehat{z}}
\def\zbar{\overline{z}}
\def\la{\leftarrow}
\def\ra{\rightarrow}
\def\MSE{\mbox{\small \sffamily MSE}}
\def\SNR{\mbox{\small \sffamily SNR}}
\def\SINR{\mbox{\small \sffamily SINR}}
\def\arr{\rightarrow}
\def\Exp{\mathbb{E}}
\def\var{\mbox{var}}
\def\Tr{\mbox{Tr}}
\def\tm1{t\! - \! 1}
\def\tp1{t\! + \! 1}

\def\Xset{{\cal X}}

\newcommand{\one}{\mathbf{1}}
\newcommand{\abf}{\mathbf{a}}
\newcommand{\bbf}{\mathbf{b}}
\newcommand{\dbf}{\mathbf{d}}
\newcommand{\ebf}{\mathbf{e}}
\newcommand{\gbf}{\mathbf{g}}
\newcommand{\hbf}{\mathbf{h}}
\newcommand{\pbf}{\mathbf{p}}
\newcommand{\pbfhat}{\widehat{\mathbf{p}}}
\newcommand{\qbf}{\mathbf{q}}
\newcommand{\qbfhat}{\widehat{\mathbf{q}}}
\newcommand{\rbf}{\mathbf{r}}
\newcommand{\rbfhat}{\widehat{\mathbf{r}}}
\newcommand{\sbf}{\mathbf{s}}
\newcommand{\sbfhat}{\widehat{\mathbf{s}}}
\newcommand{\ubf}{\mathbf{u}}
\newcommand{\ubfhat}{\widehat{\mathbf{u}}}
\newcommand{\utildebf}{\tilde{\mathbf{u}}}
\newcommand{\vbf}{\mathbf{v}}
\newcommand{\vbfhat}{\widehat{\mathbf{v}}}
\newcommand{\wbf}{\mathbf{w}}
\newcommand{\wbfhat}{\widehat{\mathbf{w}}}
\newcommand{\xbf}{\mathbf{x}}
\newcommand{\xbfhat}{\widehat{\mathbf{x}}}
\newcommand{\xbfbar}{\overline{\mathbf{x}}}
\newcommand{\ybf}{\mathbf{y}}
\newcommand{\zbf}{\mathbf{z}}
\newcommand{\zbfbar}{\overline{\mathbf{z}}}
\newcommand{\zbfhat}{\widehat{\mathbf{z}}}
\newcommand{\Ahat}{\widehat{A}}
\newcommand{\Abf}{\mathbf{A}}
\newcommand{\Bbf}{\mathbf{B}}
\newcommand{\Cbf}{\mathbf{C}}
\newcommand{\Bbfhat}{\widehat{\mathbf{B}}}
\newcommand{\Dbf}{\mathbf{D}}
\newcommand{\Ebf}{\mathbf{E}}
\newcommand{\Gbf}{\mathbf{G}}
\newcommand{\Hbf}{\mathbf{H}}
\newcommand{\Kbf}{\mathbf{K}}
\newcommand{\Pbf}{\mathbf{P}}
\newcommand{\Phat}{\widehat{P}}
\newcommand{\Qbf}{\mathbf{Q}}
\newcommand{\Rbf}{\mathbf{R}}
\newcommand{\Rhat}{\widehat{R}}
\newcommand{\Sbf}{\mathbf{S}}
\newcommand{\Ubf}{\mathbf{U}}
\newcommand{\Vbf}{\mathbf{V}}
\newcommand{\Wbf}{\mathbf{W}}
\newcommand{\Xhat}{\widehat{X}}
\newcommand{\Xbf}{\mathbf{X}}
\newcommand{\Ybf}{\mathbf{Y}}
\newcommand{\Zbf}{\mathbf{Z}}
\newcommand{\Zhat}{\widehat{Z}}
\newcommand{\Zbfhat}{\widehat{\mathbf{Z}}}
\def\alphabf{{\boldsymbol \alpha}}
\def\betabf{{\boldsymbol \beta}}
\def\mubf{{\boldsymbol \mu}}
\def\lambdabf{{\boldsymbol \lambda}}
\def\etabf{{\boldsymbol \eta}}
\def\xibf{{\boldsymbol \xi}}
\def\taubf{{\boldsymbol \tau}}
\def\sigmahat{{\widehat{\sigma}}}
\def\thetabf{{\bm{\theta}}}
\def\thetabfhat{{\widehat{\bm{\theta}}}}
\def\thetahat{{\widehat{\theta}}}
\def\mubar{\overline{\mu}}
\def\muavg{\mu}
\def\sigbf{\bm{\sigma}}
\def\etal{\emph{et al.}}
\def\Ggothic{\mathfrak{G}}
\def\Pset{{\mathcal P}}
\newcommand{\bigCond}[2]{\bigl({#1} \!\bigm\vert\! {#2} \bigr)}
\newcommand{\BigCond}[2]{\Bigl({#1} \!\Bigm\vert\! {#2} \Bigr)}

\def\Rect{\mathop{Rect}}
\def\sinc{\mathop{sinc}}
\def\Real{\mathrm{Re}}
\def\Imag{\mathrm{Im}}
\newcommand{\tran}{^{\text{\sf T}}}
\newcommand{\herm}{^{\text{\sf H}}}


% Solution environment
\definecolor{lightgray}{gray}{0.95}
\newmdenv[linecolor=white,backgroundcolor=lightgray,frametitle=Solution:]{solution}



\begin{document}

\title{Problems: Antennas and Free-Space Propagation\\
EL-GY 6023. Wireless Communications}
\author{Prof.\ Sundeep Rangan}
\date{}

\maketitle

In all the problems below, unless specified otherwise, $\phi$ is the
azimuth angle and $\theta$ is elevation angle.

\begin{enumerate}
\item \emph{EM wave}:  Suppose the E-field is,
\[
    \Ebf(x,y,z,t) = E_0 \ebf_y \cos(2\pi f t - kx).
\]
\begin{enumerate}[label=(\alph*)]
  \item What is direction of motion?
  \item If the average power flux is $10^{-8}$ mW/m$^2$, what is $E_0$?
   Assume the characteristic impedance is $\eta_0 = 377 \Omega$.
  \item If the frequency is $f=$ 1.5 GHz, what is $k$?
  What are the units of $k$?
\end{enumerate}

\item \emph{EM polarization}:  An EM plane wave has
a power flux of $10^{-8}$ mW/m$^2$ in a horizontal polarization (along the $x$-axis)
and $2(10)^{-8}$ mW/m$^2$ in a vertical polarization (along the $y$-axis).
Assume the characteristic impedance is $\eta_0 = 377 \Omega$.
\begin{enumerate}[label=(\alph*)]
  \item What is combined $E$-field value and its direction?
  \item What is the direction of motion?
  \item What is the total power flux?
\end{enumerate}


\item \emph{dBm to linear conversions:}
\begin{enumerate}[label=(\alph*)]
\item Convert the following to mW:   17 dBm, -73 dBm, -97 dBW.
\item Convert the following to dBm: 250 mW, $8(10)^{-8}$ W, $5(10)^{-6}$ mW
\end{enumerate}


\item \emph{Spherical-cartesian conversions:}
\begin{enumerate}[label=(\alph*)]
  \item Convert $(r,\phi,\theta)=(2,45^\circ,30^\circ)$ to $(x,y,z)$.
  \item Convert $(x,y,z) = (1,2,3)$ to $(r,\phi,\theta)$.
  \item Convert $(x,y,z) = (1,-2,3)$ to $(r,\phi,\theta)$.
\end{enumerate}

\item \emph{Rotation matrices}:  The \emph{rotation matrix}
$R(\theta,\phi)$ is the $3 \times 3$ matrix such that $R(\theta,\phi)\rbf$ rotates
the vector $\rbf$ by an angle pair $(\theta,\phi)$.  
You can read more about this on wikipedia or other sources.
\begin{enumerate}[label=(\alph*)]
  \item Find the entries of $R(\theta,\phi)$.
  \item Show $R(\theta,\phi)$ is \emph{orthogonal} meaning
  $R(\theta,\phi)^{-1}=R(\theta,\phi)\tran$.
  \item Is
\[
    R(\theta_1,\phi_1)R(\theta_2,\phi_2) = R(\theta_2,\phi_2)R(\theta_1,\phi_1)?
\]
That is, do the order of rotations matter?
Prove or find a counter-example.
\end{enumerate}


\item \emph{Angular areas:}  Find the angular area in steradians of
following sets of angles:
\begin{enumerate}[label=(\alph*)]
  \item $A_1 = \left\{ (\phi,\theta) ~ | ~ \phi \in [-30,30],~ \theta \in [-90,90]\right\}$
  \item $A_2 = \left\{ (\phi,\theta) ~|~ \phi \in [-30,30], ~ \theta \in [-45,45]\right\}$
\end{enumerate}

\item \emph{Directivity:}  Suppose an antenna radiates power uniformly in
the angular beam $\phi \in [-30,30], \theta \in [-45,45]$ and radiates
no power at other angles.  What is the directivity of the antenna?

\item \emph{Radiation intensity:}  A 170 cm x 40 cm object
(roughly the size of a human) is 800m from a base station.
If the antenna transmits 250~mW isotropically, how much power
reaches the human?  Use reasonable approximations that the human is far from the
transmitter.

\item \emph{Radiation integration:}  The radiation density for some antenna is,
\[
    U(\phi,\theta) = A\cos^2(\phi), \quad A = 100 \mbox{mW/sr}.
\]
\begin{enumerate}[label=(\alph*)]
\item What is the total radiated power in mW and dBm?
\item What is the directivity of the antenna in linear scale and in dBi?
\end{enumerate}

\item \emph{Numerically integrating patterns:}
Write a short MATLAB function,
\begin{lstlisting}
    function [totPow, dir] = powerDirectivity(az,el,E,eta)
\end{lstlisting}
that computes the total power and directivity as a function of the $E$-field
values.  The E field is specified as a complex matrix \mcode{E(i,j)}
at angles \mcode{az(i),el(j)}.  You can assume that the angles are uniformly
spaced over  the total angular space.   The total power output
 \mcode{totPow} should be a
scalar representing the total radiated power in dBm and the directivity output
should be \mcode{dir(i,j)} should be a matrix.

\item \emph{Friis' Law}:  A transmitter radiates 100 mW at a carrier $f_c = 2.1$ GHz
with a directional gain of $G_t = 10$ dBi.
Suppose the receiver is $d = 200$ m from the transmitter and the path is free space.
What is the received power if:
\begin{enumerate}[label=(\alph*)]
\item The effective received aperture is 1 cm$^2$.
\item The receiver gain is $G_r = 5$ dBi.
\end{enumerate}

\end{enumerate}

\end{document}



